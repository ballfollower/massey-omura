\documentclass[12pt, a4paper]{article}

\usepackage{amssymb}
\usepackage{amsmath}
\usepackage{amsthm}

\usepackage[cp1250]{inputenc}
\usepackage{polski}
\usepackage[polish, english]{babel}

\usepackage{keystroke}

\usepackage{graphicx}

\title{Massey-Omura Cipher. Project Documentation}
\author{Grzegorz K�osi�ski}

\begin{document}

\maketitle

\section*{Theory}
The Massey-Omura system is an improvement to the Shamir's three-pass protocol. The only difference is usage of different groups -- in the former case the $GF(2^m)^{*}$, for $m \in \mathbb{Z}$, and in the latter -- $GF(p)^{*}$, for $p$ a prime. The project realizes the Shamir's protocol and this one will be described.

The protocol allows two parties to pass a secret message through a public channel. We assume the passive adversary.

A one-time preparation step is required. Parties choose at random a large prime $p$, such that one cannot compute discrete logarithms mod $p$. The number is published. Then each party, \textit{A} and \textit{B}, draws a secret number, $a$ and $b$ respectively, such that $a, b \in \{1,2,\ldots,p-2\}$, $(a, p-1)=1$ and $(b, p-1)=1$. Finally, parties compute multiplicative inverses of their numbers: $a^{-1} (\text{mod~}p-1)$ and $b^{-1} (\text{mod~}p-1)$. These also have to remain secret.

The message transport works as follows. \textit{A} raises mod $p$ the message $m$ to the power $a$ and sends the result to \textit{B}. \textit{B} then raises it mod $p$ to $b$, obtaining: \[m^{ab} (\text{mod} p)\text{,}\] and sends it back to \textit{A}. \textit{A} raises it mod $p$ to $a^{-1}$, obtaining: \[m^{aba^{-1}}=m^b (\text{mod} p)\text{,}\] and sends it to \textit{B}. \textit{B} finally raises it mod $p$ to $b^{-1}$, regaining the original message: \[m^{bb^{-1}}=m (\text{mod} p)\text{.}\]

\section*{Application Interface}
The application interface is clear and easy to use. Its usage is presented in figure \ref{fig:interface}.

\begin{figure}[tbhp]
	\centering
	\includegraphics[width=1.0\textwidth]{"../graphics/screenshot commented".png}
	\caption{A basic usage scenario of the application}
	\label{fig:interface}
\end{figure}

The user clicks the "Draw" button to draw the (probable) prime $p$. (The prime's length is set by default to 64 bits, for illustration purposes, but in real-life should reach at least 1024 bits and the customization is available through a text box.) Just after $p$ is drawn, the private numbers are being drawn and displayed aside the parties.

Next, the user enters the message. After confirmation (by pressing \Enter for example), the subsequent values are computed and diplayed:

\begin{enumerate}
\item An encoded message: divided into blocks and expressed as integers
\item An A-encrypted message: raised to the power $a$
\item An B-encrypted message: raised to the power $b$
\item An A-decrypted message: raised to the power $a^{-1}$
\item An B-decrypted message: raised to the power $b^{-1}$
\item A decoded message: converted back into text

\end{enumerate}

\end{document}
